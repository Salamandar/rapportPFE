\documentclass[a4paper,12pt]{report}
\usepackage[T1]{fontenc}
\usepackage[utf8]{inputenc}
\usepackage[francais]{babel}
\usepackage[usenames,dvipsnames,svgnames,table]{xcolor}
\usepackage[colorlinks,linkcolor={blue!30!black},
                       citecolor={blue!50!black},
                        urlcolor={blue!80!black},
                        ]{hyperref}
\usepackage{titling} % Récupérer titre et auteur
\usepackage{enumerate} % Styles personnalisés d'énumération
\usepackage{amsmath,siunitx,array,url} % Symboles, etc
\usepackage{caption,subcaption,wrapfig,rotating,pdfpages} % Mise en page
\usepackage[perpage]{footmisc} % Numérotation par page des footnotes
\usepackage{epic,eepic,graphicx,tikz} % Drawings
\usepackage[top=2cm,left=2.5cm,right=2.5cm,bottom=2cm]{geometry} % Géométrie de la page, modifier selon le besoin
\setlength{\parskip}{0.4em} % Taille des interlignes entre paragraphes
\usepackage[babel=true,kerning=true]{microtype}

\addto{\captionsfrench}{\renewcommand{\abstractname}{Introduction}}
\pdfsuppresswarningpagegroup=1
\date{}
\title{Rapport de Projet de Fin d'Études : Caractérisation de pointes fibrées pour nano-pinces optiques et plasmoniques}
\author{Félix Piédallu}
\hypersetup{pdftex,
            pdfauthor=\theauthor,
            pdftitle=\thetitle}

\begin{document}
\nocite{*}
\pagenumbering{gobble}  % Pas de numérotation
\begin{titlepage}
    \vspace*{-10px}
    \includegraphics[height=80px]{Images/logo_phelma.pdf}
    \vspace*{-80px}
\begin{flushright}
    \vspace*{-10px}
    \includegraphics[height=80px]{Images/logo_neel.pdf}
\end{flushright}

\vspace*{1.5cm}
\begin{center}
\LARGE{\textsc{Félix Piédallu}}\\[1cm]

\rule{\linewidth}{0.5mm}\\[0.4cm]
{\huge{\bfseries Rapport de Projet de Fin d'Études}\\[0.4cm]
Caractérisation de pointes fibrées pour nano-pinces optiques et plasmoniques\\[0.4cm]}
\rule{\linewidth}{0.5mm}\\[0.5cm]

\large{\textsc{Filière PNS 2015-2016}}\\[2cm]

\Large{Au sein de l'équipe Nano-Optique et Forces}\\[1cm]

\Large{Sous la direction de Jochen \textsc{Fick}}\\[1cm]

\large{Du 22 Février 2016 au 22 Juillet 2016}\\[1cm]

\includegraphics[width=\textwidth]{Images/Illustration.png}\\[0cm]


\end{center}
\end{titlepage}
\newpage\null\thispagestyle{empty}\newpage
% Remerciements
\vspace*{\stretch{1}}
\begin{center}
\textsc{\Large Remerciements}
\end{center}

\vspace{0.5cm}
Je tiens tout d'abord à remercier mon maître de stage, Jochen \textsc{Fick}, qui m'a donné l'opportunité d'effectuer mon projet de fin d'études au sein de l'équipe Nano-Optique et Forces à l'Institut Néel. Je remercie également Grenoble INP Phelma pour m'avoir permis d'obtenir, à travers mes études et ce stage, une formation scientifique de qualité.
Enfin, je remercie l'ensemble des personnes avec qui j'ai pu, d'un point de vue professionnel et personnel, échanger lors de ce stage: Jean-François \textsc{Motte} et Gwénaëlle \textsc{Julie}, avec qui j'ai pu travailler en salle blanche pour la réalisation des pointes métallisées, ainsi que les membres des équipes NOF et QNES que j'ai eu la chance de côtoyer: Aurélien \textsc{Drezet}, Guillaume \textsc{Bachelier}, Herman \textsc{Sellier}, Martin \textsc{Berthel}, Nicolas \textsc{Chauvet}, Guillaume \textsc{Laurent}, Maëliss \textsc{Ethis-De-Corny}, Aline \textsc{Pham}, Vincent \textsc{Morice} et Vincent \textsc{Delmas}.

\vspace*{\stretch{3}}

\tableofcontents        % Table des matières avec liens, générée automatiquement.
\newpage
\listoffigures
\pagenumbering{arabic}  % Numérotation de retour !


\chapter*{Introduction} % Contexte du stage
\section*{Les pinces optiques}
Introduites en 1986 par A. Ashkin\cite{Ashkin}, les pinces optiques permettent de piéger et de manipuler, sans contact mécanique, des objets nanométriques. Cette particularité non destructive ni invasive les affirme comme un outil privilégié pour la manipulation et l'isolation de molécules et d'espèces biologiques.

Leur fonctionnement repose sur les forces d'interaction entre la matière et le rayonnement optique injecté par la pince. D'une part, la force de gradient attire les particules vers les maxima d'intensité lumineuse, permettant de les piéger au centre d'un faisceau, et proche de la source lumineuse. D'autre part, la force de diffusion pousse les particules dans le sens de propagation du faisceau. La conception d'une pince optique efficace et stable repose alors sur un bon équilibre entre ces deux forces.

Les pinces optiques classiques utilisent alors un faisceau laser gaussien fortement focalisé, afin de maximiser la force de gradient au point focal, et donc l'efficacité du piégeage. Mais depuis les années 2000, les pinces optiques fibrées semble une voie privilégiée pour le piégeage optique \cite{Taguchi,Taylor}. Elles ne nécessitent pas de matériel optique encombrant et peuvent être intégrées et alignées facilement.

De telles pinces optiques utilisent comme source de faisceau des fibres gravées en pointes afin d'obtenir un faisceau collimaté, très concentré, et donc des gradients d'intensité lumineuse élevés.

\begin{figure}[h] \centering
    \includegraphics[width=0.75\textwidth]{{"Images/Schemas/FaisceauConfinement"}}
    \caption{Schéma de principe du piégeage optique dans le cas de faisceaux\\ focalisé (gauche) et confiné (droite)}
    \label{cuve_eau}
\end{figure}

\section*{Contexte du stage}
Le dispositif actuel de pince optique fibrée a été développé dans le cadre de la thèse de Jean-Baptiste Decombe\cite{Decombe}. Elle est composée de deux fibres optiques monomodes, gravées à une extrémité en pointe, ce qui permet d'obtenir un faisceau Gaussien de quelques microns de largeur. L'utilisation de deux pointes fibrées permet non seulement d'avoir un confinement transversalement au faisceau grâce à la force de gradient, mais aussi selon la direction du faisceau, lorsque les forces de diffusion des deux pointes se compensent.

Cette pince optique a permis un piégeage efficace de particules diélectriques sphériques, de tailles allant du micron à quelques 60nm de diamètre.

Le but de mon stage est alors de caractériser au mieux les pointes utilisées afin de connaître au mieux leur émission, spatialement et spectralement. Cela permettra notamment de connaître correctement les différents types de pointes et leurs applications possibles.

Dans une première partie, nous allons détailler l'ensemble du dispositif expérimental que j'ai utilisé durant mon stage, et le protocole d'élaboration des pointes fibrées. Ensuite, nous verrons les caractérisations en émission spatiales et spectrales que j'ai effectuées.

\chapter{L'expérience}
Dans cette partie, nous allons détailler le dispositif expérimental qui a été mis en place afin de caractériser au mieux les pointes fibrées.

Nous verrons d'abord l'ensemble de ce qui compose l'expérience, comme l'injection du laser, les appareils utilisés pour le positionnement des pointes, ainsi que les instruments de mesures que nous avons utilisés.

Enfin, nous aborderons le processus d'élaboration des pointes fibrées et des pointes métallisées.

\section{Positionnement des pointes fibrées}
Le positionnement des pointes est effectué grâce à deux jeux de moteurs piezoélectriques, chacun permettant un déplacement dans les trois directions.
\begin{description}
    \item[Les moteurs piezoélectriques inertiels\footnotemark]\footnotetext{Mechonics MS30} utilisent le mécanisme de "stick-and-slip", en jouant sur des coefficients de frottement statique et dynamique très différents entre plateau mobile et les moteurs piezo.

    Un déplacement lent des piézos entraîne alors le plateau ("stick"), mais un déplacement rapide laisse le plateau glisser, sans bouger ("slip")." La tension appliquée sur les éléments piezoélectriques aura alors une forme de dent de scie.

    Avoir un plateau désolidarisé des éléments piezoélectriques permet d'avoir une gamme de déplacement très large (quelques cm), mais une précision faible (environ 30nm par pas). Ils permettent donc de positionner grossièrement les fibres l'une en face de l'autre.

    \item[Les platines piezoélectriques\footnotemark,]\footnotetext{Physik Instrumente PI P620} à l'inverse, ont une résolution sub-nanométrique, un positionnement absolu et un champ de déplacement réduit à 50 $\mu$m. Ils permettent alors d'aligner au mieux les pointes et d'effectuer des balayages afin d'étudier spatialement l'émission des pointes.
\end{description}

Sur chacun des jeux de moteurs piezoélectriques est fixé un support rainuré, sur lequel on peut alors fixer des fibres. Celles-ci sont fixées dans les rainures avec des plaquettes de téflon vissées sur le support. Ainsi, les fibres sont fixées parallèlement et peuvent être positionnées face à face précisément.


On peut rajouter une troisième fibre, clivée, sur un des supports, ainsi qu'un miroir sur l'autre support. La cavité de Fabri-Perrot ainsi créée permet, en injectant de la lumière blanche, de mesurer très précisément la distance relative des deux fibres: Les réfléxions interfèrent, créant ainsi des franges (fig. \ref{franges_distance}). La transformée de Fourier de ce signal nous donne directement la distance d entre le miroir et la fibre clivée (fig. \ref{schema_spectro_distance}).

Cette mesure permet alors d'asservir la distance relative entre les pointes, avec une précision de la dizaine de nm.

\begin{figure}[h] \centering
    \begin{subfigure}[c]{0.48\textwidth}
        \includegraphics[width=\textwidth]{{"Images/Schemas/SpectroDistance"}.pdf}
        \caption{Schéma de mesure de la distance}
        \label{schema_spectro_distance}
    \end{subfigure}
    \begin{subfigure}[c]{0.48\textwidth}
        \includegraphics[width=\textwidth]{{"Images/interfero"}.png}
        \caption{Franges pour différentes distances $d$}
        \label{franges_distance}
    \end{subfigure}
    \caption{Schéma de mesure de la distance par interférence (a), et franges pour différentes distances fibre-miroir (b)}
\end{figure}

Cette méthode ne nous donnant que la distance relative entre les pointes, la méthode pour obtenir la distance absolue est de rapprocher lentement les pointes jusqu'à ce qu'elles se touchent. Le contact est alors facilement visible par microscope, lorsqu'un déplacement transversal d'une pointe entraîne l'autre pointe. Une approche soigneuse permet de ne pas abîmer les pointes et d'obtenir cette distance absolue avec une précision d'environ 50nm.

\subsection{Dérive du positionnement}
Malheureusement, les moteurs piezoélectriques Mechonics ne permettent pas un positionnement absolu, et ne sont pas asservis en position: une dérive peut alors apparaître.

Une faible dérive est sans impact, mais nous avons observé pendant mon stage une vitesse de dérive jusqu'à 500 nm/minute, ce qui empêche toute mesure correcte. En effet, les mesures effectuées consistent généralement en balayages de quelques microns de large, qui durent quelques minutes.

\begin{figure}[h] \centering
    \begin{subfigure}[b]{0.43\textwidth}
        \includegraphics[width=\textwidth]{{"Images/Derive_Mecho/Avant"}.png}
        \caption{Avant mise à la masse}
        \label{mechonics_avant}
    \end{subfigure}
    \begin{subfigure}[b]{0.43\textwidth}
        \includegraphics[width=\textwidth]{{"Images/Derive_Mecho/Apres"}.png}
        \caption{Après mise à la masse}
        \label{mechonics_apres}
    \end{subfigure}
    \caption{Dérive des moteurs Mechonics avant et après mise à la masse}
\end{figure}

La dérive que nous avons pu mesurer suit une loi exponentielle, ce qui traduit une relâche mécanique des supports ou électrique au niveau du contrôleur des moteurs. Nous avons alors décidé de connecter les moteurs piezoélectriques à la masse, ce qui a grandement réduit la dérive, jusqu'à 23nm/minute selon l'axe Y comme on peut le voir sur la figure \ref{mechonics_apres}.

Néanmoins le problème de dérive reste: même avec la mise à la masse des moteurs piezoélectriques, une dérive importante est apparue quelques fois encore lors de la suite de mon stage.

\section{Chemin optique et injection Laser}
La source de lumière utilisée pour les mesures de transmission est une diode Laser émettant à $\lambda = 808$nm, ayant une puissance maximale de 250mW\footnote{Lumics LU0808-M250}. Cette diode est directement fibrée, ce qui permet l'injection d'un mode gaussien circulaire dans les fibres.

La longueur d'onde a été choisie dans l'optique de l'utilisation des pinces optiques : éloignée des bandes d'absorption de l'eau et des tissus biologiques, ainsi que du visible pour permettre le piégeage et la caractérisation de particules photo-luminescentes. Elle est aussi éloignée de la bande d'absorption de l'or qui est utilisé pour les pointes métallisées, tout en restant guidée par les fibre optiques utilisées pour fabriquer les pointes.

La figure \ref{banc_optique} présente le banc optique et l'ensemble de ses éléments.


\begin{figure}[h] \centering
    \includegraphics[width=0.9\textwidth]{{"Images/Schemas/experience"}.pdf}
    \caption{Schéma du banc optique expérimental}
    \label{banc_optique}
\end{figure}

\begin{enumerate}[{1)}]
\setcounter{enumi}{2}
    \item Les densités optiques permettent d'adapter la puissance transmise à volonté
    \item et 5) Les lames $\lambda/2$ permettent, avec le polariseur, de fixer et moduler la polarisation injectée dans les fibres.
\setcounter{enumi}{6}
    \item Le cube séparateur permet, avec un polariseur, de diviser le faisceau en deux et de régler la puissance des deux faisceaux sortants. Il est essentiellement utile pour injecter le laser dans deux pointes fibrées en même temps.
\setcounter{enumi}{8}
    \item Le micro-positionneur manuel\footnote{Thorlabs, MBT612D/M} permet, grâce à sa précision sub-micrométrique, d'aligner avec précision une fibre optique, pour y injecter le faisceau laser focalisé dans la fibre monomode, dont le mode est d'environ 4 $\mu$ m.
\setcounter{enumi}{11}
    \item et 13) Afin de limiter les manipulations d'alignement, ainsi que pour connecter directement des instruments de mesure fibrés, des connecteurs mécaniques\footnote{Corning CamSplice} sont utilisés pour connecter l'extrémité clivée des pointes optiques à des fibres clivées et connectorisées.
\end{enumerate}

\section{Les appareils de mesure et d'observation}
\subsection{Microscope optique}
Le processus d'alignement des pointes optiques permet nécessairement une observation directe des pointes.

Ceci est permis par le microscope qui a été mis en place. Il est constitué d'un objectif de microscope corrigé à l'infini\footnote{Mitutoyo} de grossissement x50 et d'ouverture numérique NA=0,55.


\section{Les appareils de mesure}
\section{L'élaboration des pointes \& des pointes métallisées}
Les pointes fibrées sont l'élément central de la pince optique fibrée. Il est donc essentiel d'avoir un processus de fabrication permettant une qualité de pointes la meilleure possible, notamment du point de vue forme. De plus, une fabrication reproductible est essentielle, ce qui nécessite un processus correctement contrôlé.

Nous profitons d'un savoir-faire déjà présent à l'Institut Néel grâce à Jean-François Motte, qui l'a développé pour la réalisation de pointes adaptées à la microscopie optique à haute résolution, telle que le SNOM (Microscope optique à balayage en champ proche) et l'iSOM (Microscope optique à balayage interférentiel).

La proximité de la fabrication des pointes nous permet d'adapter le processus en fonction de nos besoins, notamment pour la fabrication de pointes métallisées comme nous allons le voir.

\subsection{Fabrication des pointes fibrées}
La fibre optique utilisée pour la fabrication des pointes est une fibre monomode entre 630nm et 800nm à saut d'indice\footnote{Nufern S630-HP}. Le cœur de la fibre est en silice pure, a un indice de 1.46 et un diamètre de $3,5\mu m$. La lumière est guidée dans ce cœur par la gaine optique d'indice de réfraction légèrement plus faible, tandis qu'une gaine plastique mécanique protège la fibre et la rend flexible.

Différents processus existent pour la réalisation de fibres optiques effilées, notamment des processus d'étirage et de polissage mécaniques, mais aussi de gravure chimique. 

Cette dernière méthode, qui est maîtrisée à l'Institut Néel, permet d'obtenir des pointes d'angle assez grand et constant; en outre, elle ne modifie pas les caractéristiques du cœur de la fibre, contrairement aux processus d'étirage mécanique. Néanmoins, la forme de la fibre n'est pas toujours conique mais souvent aplatie, ce qui peut parfois poser problème pour la fabrication de pointes métalliques comme nous pourrons le voir.

La fibre optique est immergée dans une solution d'acide fluorhydrique (HF à 40\%). Cet acide dissout très facilement la silice, mais n'attaque pas la gaine plastique qui l'entoure. Ainsi, un ménisque concave se forme sur les parois de la gaine. Ce ménisque remonte au fur et à mesure que la pointe devient conique, jusqu'à obtenir une pointe effilée comme on peut le voir sur la figure \ref{gravure_chimique}.

Ce processus permet d'obtenir une pointe dont l'extrémité se trouve sur l'axe de la fibre et a un diamètre d'environ 60nm; L'angle du cône ainsi formé est constant, d'environ 15\si{\degree}.

\begin{figure}[h]\centering
    \begin{subfigure}[b]{0.45\textwidth}\label{l33fm2_meb_avant}
        \includegraphics[width=\textwidth]{{"Images_pasmoi/gravure"}}
    \end{subfigure}
    \begin{subfigure}[b]{0.50\textwidth}\label{l33fm2_meb}
        \includegraphics[width=\textwidth]{{"Images_pasmoi/gravure_2"}}
    \end{subfigure}
    \caption{Processus de gravure chimique des pointes fibrées}
    \label{gravure_chimique}
\end{figure}

La manipulation d'acide fluorhydrique nécessite de nombreuses précautions, en raison de sa forte toxicité. C'est pourquoi le bain d'acide est recouvert d'une couche d'huile de silicone, afin d'empêcher des vapeurs d'acide fluorhydrique de se former.

\subsection{Fabrication des pointes métallisées}
Les pointes que nous avons ainsi obtenues peuvent être utilisées directement, ou être métallisées afin d'obtenir une pointe d'ouverture très petite. Pour cela nous recouvrons intégralement la pointe de métal, puis nous ouvrons son extrémité grâce à un faisceau d'ions focalisé.


\subsubsection*{Métallisation des pointes}
Les pointes sont métallisées par évaporation thermique. Elles sont d'abord recouvertes d'une fine couche d'accroche de Titane (Ti) de quelques nanomètres, sur lequel on dépose la couche d'or (Au) dont l'épaisseur peut varier entre 40nm et 150nm.

La chambre d'évaporation est placée sous un vide de l'ordre de $10^{-7}$mbar; le métal à déposer est chauffé dans un creuset par effet Joule et se recondense sur les pointes placées au-dessus (fig. \ref{metallisation}).

Afin d'obtenir une couche uniforme de métal, les pointes sont tournées à vitesse constante. L'épaisseur déposée peut être finement contrôlée à l'aide d'une balance à quartz.
\begin{figure}[h]\centering
    \includegraphics[width=0.6\textwidth]{{"Images_pasmoi/metallisation_2"}}
    \caption{Dépôt de métal sur les pointes par évaporation thermique}
    \label{metallisation}
\end{figure}

Il faut néanmoins faire attention à ne pas déposer une épaisseur trop faible d'or. En effet, l'épaisseur de la couche déposée est variable et peut localement s'approcher de l'épaisseur de peau de l'or, qui est d'environ $14nm$ dans l'infrarouge proche. Au-deçà de $50nm$, on peut obtenir une pointe partiellement transparente à la lumière, et donc inutilisable.

\subsubsection*{Découpe de l'extrémité des pointes}
L'extrémité des pointes métallisées est alors complètement recouverte d'or. Nous utilisons alors le FIB (Faisceau d'ions focalisé) afin d'ouvrir cette extrémité (fig. \ref{fib_avant_apres}).

Les pointes sont placées dans un MEB sous ultra-vide, à 90\si\degree du faisceau d'ions. L'utilisation en parallèle du MEB et du FIB permet de contrôler en temps réel la découpe de la pointe, on peut donc obtenir une ouverture de la taille précise voulue.


\begin{figure}[h]\centering
    \begin{subfigure}[b]{0.25\textwidth}\label{l33fm2_meb_avant}
        \includegraphics[width=\textwidth]{{"Images/Photos_Fibres/l33fm2_avant_crop"}}
    \end{subfigure}
    \begin{subfigure}[b]{0.25\textwidth}\label{l33fm2_meb}
        \includegraphics[width=\textwidth]{{"Images/Photos_Fibres/l33fm2_apres_crop"}}
    \end{subfigure}
    \caption{Pointe métallisée avant et après découpe au FIB}
    \label{fib_avant_apres}
\end{figure}

Cette méthode permet d'obtenir une ouverture plane dont la taille peut être contrôlée avec une précision nanométrique.

On peut remarquer sur la figure \ref{fib_avant_apres} que l'ouverture de la pointe n'est pas circulaire mais très allongée (de grand et petit axes 200nm$\times$40nm). En effet le processus de gravure chimique n'a pas permis d'obtenir une pointe conique mais plutôt aplatie.

\subsection{Images FIB}

L'excentricité de l'apex d'une pointe de petit axe $a$ et grand axe $b$ est définie par 
\[e=\frac{\sqrt{a^2-b^2}}{a}\]
\begin{figure}[h]\centering
    \begin{subfigure}[b]{0.35\textwidth}\label{l27fm5_meb}
        \includegraphics[width=\textwidth]{{"Images/Photos_Fibres/l27fm5_ronde_crop"}.png}
        \caption*{Apex circulaire ($e = 0,28$)}
    \end{subfigure}
    \begin{subfigure}[b]{0.35\textwidth}\label{l40fm3_meb}
        \includegraphics[width=\textwidth]{{"Images/Photos_Fibres/l40fm3_dims_crop_2"}.png}
        \caption*{Apex allongé ($e = 0,98$)}
    \end{subfigure}
    \caption{Clichés MEB de pointes coupées au FIB}
    \label{fib_ronde_allongee}
\end{figure}


\section{Couplage dans la fibre multimode}
Dummy text











\chapter{Caractérisation des pointes}
Nous allons détailler dans ce chapitre les différentes méthodes de caractérisation des pointes, afin de mettre en valeur les propriétés des différentes pointes qui ont été étudiées.

Dans un premier temps nous allons observer l'émission des pointes en fonction de la polarisation de la lumière injectée dans la fibre. \par

Ensuite, nous allons mesurer l'évolution de la transmission des pointes en fonction de la distance entre la pointe émettrice et la pointe de mesure. Ceci nous permettra de comprendre la structure du spot émis, et d'en conclure sur leur utilisabilité dans le cadre des nano-pinces optiques. \par

Enfin, nous allons étudier le spectre de transmission d'une telle pince, afin de mettre en valeur la transmission des différents éléments de la pince optique.

\section{Variation de la polarisation de la lumière incidente}
Il avait été vu dans de précédentes études que la transmission des pointes métallisées dépendait fortement de la polarisation de la lumière incidente \cite{Decombe}. Notamment, cette dépendance doit se retrouver essentiellement pour des apex elliptiques.
Les modélisations qui avaient été effectuées peuvent être retrouvées dans la figure \ref{theorie_polar_metal}.
\begin{figure}[h]\centering
    \includegraphics[width=0.15\textwidth]{{"Images_pasmoi/Polar_0"}.png}\hspace*{5mm}
    \includegraphics[width=0.15\textwidth]{{"Images_pasmoi/Polar_32"}.png}\hspace*{5mm}
    \includegraphics[width=0.15\textwidth]{{"Images_pasmoi/Polar_90"}.png}
    \caption{Intensité calculée avec trois polarisations incidentes caractéristiques \cite{Decombe}}
    \label{theorie_polar_metal}
\end{figure}

\subsection{Pointes sans métal}
Nous avons d'abord décidé de vérifier que les pointes non métallisées n'ont pas de dépendance en polarisation.

La figure \ref{var_lambda_nues} présente les profils d'émission d'une pointe non métallisée à deux polarisations incidentes orthogonales. On constate facilement que la polarisation de la lumière incidente n'a pas d'impact sur l'émission: seule une petite variation d'intensité traduit les irrégularités des pointes utilisées.
\begin{figure}
    \begin{center}
        \includegraphics[width=.5\textwidth]{Images/Photos_Fibres/Camera/Pointes_nues_air_crop.png}
        \caption{Deux pointes non métallisées}
        \label{pointes_nues_photo}
    \end{center}
\end{figure}

\begin{figure}[h]\centering
    \begin{subfigure}[b]{0.15\textwidth}
        \includegraphics[width=\textwidth]{{"Images/Var_Lambda/Nues_2/0deg"}.png}
        \caption{$\lambda$ = 0\si{\degree}}
    \end{subfigure}
    \begin{subfigure}[b]{0.15\textwidth}
        \includegraphics[width=\textwidth]{{"Images/Var_Lambda/Nues_2/45deg"}.png}
        \caption{$\lambda$ = 90\si{\degree}}
    \end{subfigure}
    \caption{Influence de la polarisation sur la transmission de pointes non métallisées}
    \label{var_lambda_nues}
\end{figure}

Ce résultat nous permet notamment de nous assurer de la validité des prochaines mesures: (TODO trouver une formulation)

\subsection{Pointes métallisées}
Nous avons donc ensuite observé l'effet de la polarisation de la lumière incidente sur l'émission d'une pointe métallisée. Nous avons utilisé une fibre non métallisée pour mesurer le profil d'émission d'une pointe métallisée, afin de ne mesurer qu'un effet dû à cette pointe métallisée.

La figure \ref{var_lambda_metal} présente les résultats obtenus avec une pointe très allongée (fig. \ref{pointe_metal_var_lambda}), dans l'air, avec une distance entre les pointes de l'ordre du micron.

\begin{figure}[h]\centering
    \includegraphics[width=0.2\textwidth]{{"Images/Photos_Fibres/l40fm3_dims_crop_2"}.png}
    \caption{Pointe allongée (excentricité $= 0,98$)}\vspace*{5mm}
    \label{pointe_metal_var_lambda}

    \begin{subfigure}[b]{0.136\textwidth}
        \includegraphics[width=\textwidth]{{"Images/Var_Lambda/Metal_2/32d"}.png}
        \caption*{$\lambda$ = 0\si{\degree}}
    \end{subfigure}
    \begin{subfigure}[b]{0.136\textwidth}
        \includegraphics[width=\textwidth]{{"Images/Var_Lambda/Metal_2/47d"}.png}
        \caption*{$\lambda$ = 30\si{\degree}}
    \end{subfigure}
    \begin{subfigure}[b]{0.136\textwidth}
        \includegraphics[width=\textwidth]{{"Images/Var_Lambda/Metal_2/62d"}.png}
        \caption*{$\lambda$ = 60\si{\degree}}
    \end{subfigure}
    \begin{subfigure}[b]{0.136\textwidth}
        \includegraphics[width=\textwidth]{{"Images/Var_Lambda/Metal_2/77d_2"}.png}
        \caption*{$\lambda$ = 90\si{\degree}}
    \end{subfigure}
    \begin{subfigure}[b]{0.136\textwidth}
        \includegraphics[width=\textwidth]{{"Images/Var_Lambda/Metal_2/92d"}.png}
        \caption*{$\lambda$ = 120\si{\degree}}
    \end{subfigure}
    \begin{subfigure}[b]{0.136\textwidth}
        \includegraphics[width=\textwidth]{{"Images/Var_Lambda/Metal_2/107d"}.png}
        \caption*{$\lambda$ = 150\si{\degree}}
    \end{subfigure}
    \begin{subfigure}[b]{0.136\textwidth}
        \includegraphics[width=\textwidth]{{"Images/Var_Lambda/Metal_2/122d_3"}.png}
        \caption*{$\lambda$ = 180\si{\degree}}
    \end{subfigure}
    \caption{Profils d'émission en fonction de la polarisation incidente}
    \label{var_lambda_metal}
\end{figure}

On peut tout d'abord noter que les spots semblent beaucoup plus larges ($\sim 4\mu m$) que l'apex de la pointe métallisée, qui est de l'ordre de 200nm. Nous mesurons en effet, comme nous l'avons vu dans la première partie (TODO), la convolution entre la fibre émettrice et réceptrice ; La fibre réceptrice, non métallisée, a une grande ouverture optique, ce qui donne de telles mesures.

Enfin, on remarque bien une dépendance en polarisation telle qu'on l'attendait: pour $\lambda = 0\si{\degree}$, il apparaît deux spots assez éloignés, la lumière incidente est alors polarisée selon le petit axe de la pointe métallisée ; tandis que pour  $\lambda = 90\si{\degree}$, on ne peut distinguer qu'un seul lobe: on ne distingue pas les deux spots prévus par la simulation.

Ces mesures nous permettent notamment de conclure que le grand axe de notre pointe métallisée est orienté horizontalement selon les profils d'émission.

%TODO mettre ça dans la première partie %De telles mesures ont donc été effectuées, en faisant varier la polarisation injectée grâce à un polariseur (qui permet de filtrer une seule polarisation) et une lame $\lambda/2$ (qui permet de faire varier cette polarisation).

%TODO: utiliser 2 pointes applaties ? Déjà fait, p. 30… Pas très concluant.
%On peut alors observer dans les figures \ref{var_lambda_200nm} et \ref{var_lambda_3um} les spectres selon la polarisation. La pointe métalisée est relativement ronde, ce qui se traduit par une assez faible dépendance en polarisation. On observe tout de fois un déplacement du spot principal d'émission entre $\lambda/2 = 0^o$ et $\lambda/2 = 45^o$.
% \begin{figure}\centering
%         \includegraphics[width=0.3\textwidth]{{"Images/Photos_Fibres/l27fm5_ronde"}.png}
%         \caption{Pointe métallisée presque circulaire (peu de dépendance en polarisation)}
%         \label{l27fm5_photo}
% \end{figure}



\subsection{Pointes de Bessel}
Nous avons effectué les mêmes mesures avec une pointe de Bessel. Nous attendons le même résultat que pour les fibres non métallisées: pas, ou très peu, de dépendance en polarisation incidente.

TODO rapide explication théorique ?

Nous avons alors utilisé une fibre non métallisée pour étudier l'émission, dans l'air, d'une pointe de Bessel. On constate bien sur la figure \ref{var_lambda_bessel} que la polarisation n'a pas d'incidence sur l'émission mesurée, cela confirme nos attentes.

\begin{figure}[h]\centering
    \includegraphics[width=0.4\textwidth]{{"Images/Photos_Fibres/Camera/Bessel_1"}.png}
    \caption{Photo au microscope d'une pointe de Bessel}\vspace*{5mm}
    \label{pointe_bessel_var_lambda}

    \begin{subfigure}[b]{0.136\textwidth}
        \includegraphics[width=\textwidth]{{"Images/Bessel/VarLambda/0d"}.png}
        \caption*{$\lambda$ = 0\si{\degree}}
    \end{subfigure}
    \begin{subfigure}[b]{0.136\textwidth}
        \includegraphics[width=\textwidth]{{"Images/Bessel/VarLambda/15d"}.png}
        \caption*{$\lambda$ = 30\si{\degree}}
    \end{subfigure}
    \begin{subfigure}[b]{0.136\textwidth}
        \includegraphics[width=\textwidth]{{"Images/Bessel/VarLambda/30d"}.png}
        \caption*{$\lambda$ = 60\si{\degree}}
    \end{subfigure}
    \begin{subfigure}[b]{0.136\textwidth}
        \includegraphics[width=\textwidth]{{"Images/Bessel/VarLambda/45d"}.png}
        \caption*{$\lambda$ = 90\si{\degree}}
    \end{subfigure}
    \begin{subfigure}[b]{0.136\textwidth}
        \includegraphics[width=\textwidth]{{"Images/Bessel/VarLambda/60d"}.png}
        \caption*{$\lambda$ = 120\si{\degree}}
    \end{subfigure}
    \begin{subfigure}[b]{0.136\textwidth}
        \includegraphics[width=\textwidth]{{"Images/Bessel/VarLambda/75d"}.png}
        \caption*{$\lambda$ = 150\si{\degree}}
    \end{subfigure}
    \begin{subfigure}[b]{0.136\textwidth}
        \includegraphics[width=\textwidth]{{"Images/Bessel/VarLambda/90d"}.png}
        \caption*{$\lambda$ = 180\si{\degree}}
    \end{subfigure}
    \caption{Profils d'émission en fonction de la polarisation incidente}
    \label{var_lambda_bessel}
\end{figure}


\section{Variation de la distance}
Une caractéristique importante des pointes est l'évolution du spot d'émission en fonction de la distance. En effet, cela détermine directement les performance du piégeage optique en fonction de la distance entre les pointes qui forment la pince optique.

Pour cela, nous mesurons les profils d'émission des pointes, en éloignant la pointe de mesure de la pointe émettrice. Nous obtenons alors des spots gaussiens, dont la largeur évolue en fonction de la distance.

Dans un premier temps, nous avons caractérisé les pointes dans l'air, puis dans l'eau afin de se rapprocher des conditions expérimentales des pinces optiques.

\subsection{Pointes sans métal}
On peut facilement reconnaître des fibres de bonne et mauvaise qualité: une pointe correcte présente un spot d'émission gaussien, quelle que soit la distance de mesure. La figure \ref{fibre_ok_scan} présente des exemples de scans caractéristiques.

\begin{figure}[h]\centering
    \begin{subfigure}[b]{0.136\textwidth}
        \includegraphics[width=\textwidth]{{"Images/Vrac/Nue_OK_1"}.png}
        \caption{}
    \end{subfigure}
    \begin{subfigure}[b]{0.136\textwidth}
        \includegraphics[width=\textwidth]{{"Images/Vrac/Nue_Pasok"}.png}
        \caption{}
    \end{subfigure}
    \caption{Exemples de profils de transmission caractéristiques pour une bonne (a) et une mauvaise (b) pointes}
    \label{fibre_ok_scan}
\end{figure}


\begin{figure}[h]\centering
TODO inclure des scans + une courbe d'évolution dans l'air et dans l'eau
    \caption{Profils d'émission dans l'air (-) et dans l'eau (-) et évolution du waist avec la distance ()}
    \label{nues_distance}
\end{figure}

La largeur des spots évolue linéairement avec la distance dans l'air et dans l'eau, avec une largeur au contact autour de $nm$ dans les deux milieux.
On obtient l'angle d'émission de la pointe à partir de la pente du waist: 
\[
    \theta_{\text{émission}} = \arctan(\text{pente})
\]
Et l'ouverture numérique de la pointe est obtenue par:
\[
    N.A = n\sin(\frac{\theta}{2})
\]

Dans l'air, on trouve un angle d'émission de $18\si{\degree}$, qui donne une ouverture numérique de 0,16.

Dans l'eau, le saut d'indice entre la fibre (n=1,45) et l'eau (1,33) beaucoup étant moins important qu'avec l'air (n=1), le faisceau diverge beaucoup moins: on obtient un angle d'émission de $8\si{\degree}$ et une ouverture numérique de 0,09.

Enfin, l'amplitude de l'intensité lumineuse transmise évolue selon $\dfrac{1}{\sqrt{\text{distance}}}$: en effet le flux total transmis entre les deux pointes reste constant tandis que la surface de transmission évolue quadratiquement.

Cette intensité est trois fois plus importante dans l'eau que dans l'air: cela implique un facteur $\sqrt{3}$ en puissance.


\subsection{Pointes métalliques}
Nous avons alors fait la même étude pour des pointes métalliques. Ici nous avons pris une pointe la plus circulaire possible, pour limiter l'impact de la polarisation incidente sur l'émission.

\begin{figure}[h]\centering
    \begin{subfigure}[b]{0.156\textwidth}
        \includegraphics[width=\textwidth]{{"Images/Metal/1/d=0.1"}.png}
        \caption*{100nm}
    \end{subfigure}
    \begin{subfigure}[b]{0.156\textwidth}
        \includegraphics[width=\textwidth]{{"Images/Metal/1/d=0.2"}.png}
        \caption*{200nm}
    \end{subfigure}
    \begin{subfigure}[b]{0.156\textwidth}
        \includegraphics[width=\textwidth]{{"Images/Metal/1/d=0.3"}.png}
        \caption*{300nm}
    \end{subfigure}
    \begin{subfigure}[b]{0.156\textwidth}
        \includegraphics[width=\textwidth]{{"Images/Metal/1/d=0.4"}.png}
        \caption*{400nm}
    \end{subfigure}
    \begin{subfigure}[b]{0.156\textwidth}
        \includegraphics[width=\textwidth]{{"Images/Metal/1/d=0.5"}.png}
        \caption*{500nm}
    \end{subfigure}
    \begin{subfigure}[b]{0.156\textwidth}
        \includegraphics[width=\textwidth]{{"Images/Metal/1/d=1"}.png}
        \caption*{1$\mu$m}
    \end{subfigure}
    \begin{subfigure}[b]{0.156\textwidth}
        \includegraphics[width=\textwidth]{{"Images/Metal/1/d=2"}.png}
        \caption*{2$\mu$m}
    \end{subfigure}
    \begin{subfigure}[b]{0.156\textwidth}
        \includegraphics[width=\textwidth]{{"Images/Metal/1/d=3"}.png}
        \caption*{3$\mu$m}
    \end{subfigure}
    \begin{subfigure}[b]{0.156\textwidth}
        \includegraphics[width=\textwidth]{{"Images/Metal/1/d=4"}.png}
        \caption*{4$\mu$m}
    \end{subfigure}
    \begin{subfigure}[b]{0.156\textwidth}
        \includegraphics[width=\textwidth]{{"Images/Metal/1/d=5"}.png}
        \caption*{5$\mu$m}
    \end{subfigure}
    \begin{subfigure}[b]{0.156\textwidth}
        \includegraphics[width=\textwidth]{{"Images/Metal/1/d=6"}.png}
        \caption*{6$\mu$m}
    \end{subfigure}
    \begin{subfigure}[b]{0.156\textwidth}
        \includegraphics[width=\textwidth]{{"Images/Metal/1/d=7"}.png}
        \caption*{7$\mu$m}
    \end{subfigure}
    \begin{subfigure}[b]{0.156\textwidth}
        \includegraphics[width=\textwidth]{{"Images/Metal/1/d=8"}.png}
        \caption*{8$\mu$m}
    \end{subfigure}
    \begin{subfigure}[b]{0.156\textwidth}
        \includegraphics[width=\textwidth]{{"Images/Metal/1/d=9"}.png}
        \caption*{9$\mu$m}
    \end{subfigure}
    \begin{subfigure}[b]{0.156\textwidth}
        \includegraphics[width=\textwidth]{{"Images/Metal/1/d=10"}.png}
        \caption*{10$\mu$m}
    \end{subfigure}
    \caption{Émission d'une pointe métallique en fonction de la distance entre les pointes}
    \label{emission_metal_distance}
\end{figure}


\subsection{Pointes de Bessel}
\begin{figure}[h]\centering
    \begin{subfigure}[b]{0.136\textwidth}
        \includegraphics[width=\textwidth]{{"Images/Bessel/VarDistance/fS2_d=4.9"}.png}
        \caption*{d=4.9$\mu$m}
    \end{subfigure}
    \begin{subfigure}[b]{0.136\textwidth}
        \includegraphics[width=\textwidth]{{"Images/Bessel/VarDistance/fS2_d=20.1"}.png}
        \caption*{d=20$\mu$m}
    \end{subfigure}
    \begin{subfigure}[b]{0.136\textwidth}
        \includegraphics[width=\textwidth]{{"Images/Bessel/VarDistance/fS2_d=34.6"}.png}
        \caption*{d=35$\mu$m}
    \end{subfigure}
    \begin{subfigure}[b]{0.136\textwidth}
        \includegraphics[width=\textwidth]{{"Images/Bessel/VarDistance/fS2_d=62"}.png}
        \caption*{d=62$\mu$m}
    \end{subfigure}
    \begin{subfigure}[b]{0.136\textwidth}
        \includegraphics[width=\textwidth]{{"Images/Bessel/VarDistance/fS2_d=75"}.png}
        \caption*{d=75$\mu$m}
    \end{subfigure}
    \begin{subfigure}[b]{0.136\textwidth}
        \includegraphics[width=\textwidth]{{"Images/Bessel/VarDistance/fS2_d=88"}.png}
        \caption*{d=88$\mu$m}
    \end{subfigure}
    \begin{subfigure}[b]{0.136\textwidth}
        \includegraphics[width=\textwidth]{{"Images/Bessel/VarDistance/fS2_d=101"}.png}
        \caption*{d=101$\mu$m}
    \end{subfigure}\\
    \begin{subfigure}[b]{0.136\textwidth}
        \includegraphics[width=\textwidth]{{"Images/Bessel/VarDistance/fS2_d=114"}.png}
        \caption*{d=114$\mu$m}
    \end{subfigure}
    \begin{subfigure}[b]{0.136\textwidth}
        \includegraphics[width=\textwidth]{{"Images/Bessel/VarDistance/fS2_d=126"}.png}
        \caption*{d=126$\mu$m}
    \end{subfigure}
    \begin{subfigure}[b]{0.136\textwidth}
        \includegraphics[width=\textwidth]{{"Images/Bessel/VarDistance/fS2_d=139"}.png}
        \caption*{d=139$\mu$m}
    \end{subfigure}
    \begin{subfigure}[b]{0.136\textwidth}
        \includegraphics[width=\textwidth]{{"Images/Bessel/VarDistance/fS2_d=150"}.png}
        \caption*{d=150$\mu$m}
    \end{subfigure}
    \begin{subfigure}[b]{0.136\textwidth}
        \includegraphics[width=\textwidth]{{"Images/Bessel/VarDistance/fS2_d=173"}.png}
        \caption*{d=173$\mu$m}
    \end{subfigure}
    \begin{subfigure}[b]{0.136\textwidth}
        \includegraphics[width=\textwidth]{{"Images/Bessel/VarDistance/fS2_d=197"}.png}
        \caption*{d=197$\mu$m}
    \end{subfigure}
    \begin{subfigure}[b]{0.136\textwidth}
        \includegraphics[width=\textwidth]{{"Images/Bessel/VarDistance/fS2_d=221"}.png}
        \caption*{d=221$\mu$m}
    \end{subfigure}
    \caption{Variation de l'émission d'une pointe de Bessel en fonction de la distance (TODO faire un choix dans les images)}
    \label{var_distance_bessel}
\end{figure}

\begin{figure}[h]
    \begin{center}
        \includegraphics[width=0.7\textwidth]{{"Images/Bessel/VarDistance/waist2"}.png}
        \caption{Évolution de la largeur du spot d'émission d'une pointe de Bessel}
        \label{var_distance_bessel_courbe}
    \end{center}
\end{figure}

\subsection{Franges d'émission des pointes de Bessel}
Dummy text
\begin{figure}[h]\centering
    \begin{subfigure}[b]{0.156\textwidth}
        \includegraphics[width=\textwidth]{{"Images/Bessel/Franges/d=5"}.png}
        \caption*{5$\mu$m}
    \end{subfigure}
    \begin{subfigure}[b]{0.156\textwidth}
        \includegraphics[width=\textwidth]{{"Images/Bessel/Franges/d=6"}.png}
        \caption*{6$\mu$m}
    \end{subfigure}
    \begin{subfigure}[b]{0.156\textwidth}
        \includegraphics[width=\textwidth]{{"Images/Bessel/Franges/d=7"}.png}
        \caption*{7$\mu$m}
    \end{subfigure}
    \begin{subfigure}[b]{0.156\textwidth}
        \includegraphics[width=\textwidth]{{"Images/Bessel/Franges/d=8"}.png}
        \caption*{8$\mu$m}
    \end{subfigure}
    \begin{subfigure}[b]{0.156\textwidth}
        \includegraphics[width=\textwidth]{{"Images/Bessel/Franges/d=9"}.png}
        \caption*{9$\mu$m}
    \end{subfigure}
    \begin{subfigure}[b]{0.156\textwidth}
        \includegraphics[width=\textwidth]{{"Images/Bessel/Franges/d=10"}.png}
        \caption*{10$\mu$m}
    \end{subfigure}
    \begin{subfigure}[b]{0.156\textwidth}
        \includegraphics[width=\textwidth]{{"Images/Bessel/Franges/d=12"}.png}
        \caption*{12$\mu$m}
    \end{subfigure}
    \begin{subfigure}[b]{0.156\textwidth}
        \includegraphics[width=\textwidth]{{"Images/Bessel/Franges/d=14"}.png}
        \caption*{14$\mu$m}
    \end{subfigure}
    \begin{subfigure}[b]{0.156\textwidth}
        \includegraphics[width=\textwidth]{{"Images/Bessel/Franges/d=16"}.png}
        \caption*{16$\mu$m}
    \end{subfigure}
    \begin{subfigure}[b]{0.156\textwidth}
        \includegraphics[width=\textwidth]{{"Images/Bessel/Franges/d=18"}.png}
        \caption*{18$\mu$m}
    \end{subfigure}
    \begin{subfigure}[b]{0.156\textwidth}
        \includegraphics[width=\textwidth]{{"Images/Bessel/Franges/d=20"}.png}
        \caption*{20$\mu$m}
    \end{subfigure}
    \begin{subfigure}[b]{0.156\textwidth}
        \includegraphics[width=\textwidth]{{"Images/Bessel/Franges/d=25"}.png}
        \caption*{25$\mu$m}
    \end{subfigure}
    \begin{subfigure}[b]{0.156\textwidth}
        \includegraphics[width=\textwidth]{{"Images/Bessel/Franges/d=30"}.png}
        \caption*{30$\mu$m}
    \end{subfigure}
    \begin{subfigure}[b]{0.156\textwidth}
        \includegraphics[width=\textwidth]{{"Images/Bessel/Franges/d=35"}.png}
        \caption*{35$\mu$m}
    \end{subfigure}
    \begin{subfigure}[b]{0.156\textwidth}
        \includegraphics[width=\textwidth]{{"Images/Bessel/Franges/d=40"}.png}
        \caption*{40$\mu$m}
    \end{subfigure}
    \begin{subfigure}[b]{0.156\textwidth}
        \includegraphics[width=\textwidth]{{"Images/Bessel/Franges/d=45"}.png}
        \caption*{45$\mu$m}
    \end{subfigure}
    \begin{subfigure}[b]{0.156\textwidth}
        \includegraphics[width=\textwidth]{{"Images/Bessel/Franges/d=50"}.png}
        \caption*{50$\mu$m}
    \end{subfigure}
    \caption{Franges de Bessel en fonction de la distance (TODO faire un choix dans les images)}
    \label{franges_bessel}
\end{figure}

\begin{figure}[h]
    \begin{center}
        \includegraphics[width=0.7\textwidth]{{"Images/Bessel/Franges/Franges"}.png}
        \caption{Rayon des franges de Bessel en fonction de la distance}
        \label{franges_bessel_courbe}
    \end{center}
\end{figure}





\section{Transmission en spectre}
\subsection{Normalisation}
\begin{figure}[h]
    \begin{center}
        \includegraphics[width=0.47\textwidth]{{"Images/Spectro/NormalisationIntensite"}.png}
        \includegraphics[width=0.47\textwidth]{{"Images/Spectro/NormalisationTransmission"}.png}
        \caption{Courbes de normalisation en intensité et en transmission}
    \end{center}
        \label{normalisation_transmission}
\end{figure}

\subsection{Comparaison des fibres}

\begin{figure}[h]
    \begin{center}
        \includegraphics[width=0.47\textwidth]{{"Images/Spectro/FibresNues"}.png}\\
        \includegraphics[width=0.47\textwidth]{{"Images/Spectro/FibresMetal"}.png}
        \caption{Transmission des pointes nues et métallisées, dans l'eau et dans l'air}
    \end{center}
        \label{pointes_transmission}
\end{figure}



Dummy text


\chapter*{Bilan}
\addcontentsline{toc}{chapter}{Bilan}
%\input{}



\bibliographystyle{unsrt}
\bibliography{bibliographie}
\newpage
\appendix
\phantomsection
\addcontentsline{toc}{part}{Annexes}

\vspace*{8cm}
\begin{center}
\rule{\linewidth}{0.5mm}\\[0.7cm]
{\huge{\bfseries Annexes}}\\[0.4cm]
\rule{\linewidth}{0.5mm}\\[0.5cm]


\end{center}
%\newpage
%\phantomsection
%\addcontentsline{toc}{chapter}{Guide de câblage du cryostat}
%\includepdf[pages={-}]{../guide.pdf}

\newpage
%\input{9.Abstract}

\end{document}
